\documentclass[11pt,towside]{article}
\usepackage[brazilian]{babel}			% Ajusta a língua para Portugês Brasileiro
\usepackage{amsmath}					% Permite escrever fórmulas matemáticas
\usepackage[utf8]{inputenc}			% Lê o arquivo fonte codificado em UTF-8
\usepackage[T1]{fontenc}				% Codifica os tipos no arquivo de saída usando T1
\usepackage[a4paper]{geometry}			% Ajusta o tamanho do papel para A4
\usepackage{titling}					% Permite fazer ajustes no título do artigo
\usepackage{fancyhdr}					% Permite ajustar cabeçalhos e rodapés
\usepackage{indentfirst}				% Ajusta a indentação do primeiro parágrafo
\usepackage{pgfplotstable}
\usepackage{booktabs}
\usepackage{graphicx}
\usepackage[section]{placeins}
\usepackage{float}
\usepackage{listings}
\usepackage{color}
\usepackage{imakeidx}
\makeindex[columns=3, title=Alphabetical Index, intoc]

\definecolor{dkgreen}{rgb}{0,0.6,0}
\definecolor{gray}{rgb}{0.5,0.5,0.5}
\definecolor{mauve}{rgb}{0.58,0,0.82}

\lstset{frame=tb,
  language=C,
  aboveskip=3mm,
  belowskip=3mm,
  showstringspaces=false,
  columns=flexible,
  basicstyle={\small\ttfamily},
  numbers=left,
  numberstyle=\color{gray},
  keywordstyle=\color{blue},
  commentstyle=\color{dkgreen},
  stringstyle=\color{mauve},
  breaklines=fasle,
  breakatwhitespace=true,
  tabsize=3,
  frame=L
}


\graphicspath{{img/}}

%% Define estilo das tabelas %%
\pgfplotstableset {
	every head row/.style={before row=\toprule,after row=\midrule},
    every last row/.style={after row=\bottomrule}
}
            
%% Inicio do documento %%
\usepackage{amsmath}
\begin{document}

%% Título do artigo %%
\begin{titlepage}
\begin{center}

\textsc{\Huge Universidade de São Paulo}\\[5mm]
\textsc{\Large Análise de Desempenho Concorrente Para o Método de Jacobi-Richardson}

\vfill
{\Huge SSC0143 -- Programação Concorrente}
\vfill

\begin{tabular}{rl}
\emph{Professor:}& {\Large Júlio Cezar Estrella} \\[5mm]
\emph{Alunos:}& {\Large Luz Padilla} \\[2mm]
& {\Large Paulo B. N. Nascimento}
\end{tabular}\\[25mm]

\end{center}
\end{titlepage}

\tableofcontents
\vspace{1cm}
\hrule
\vspace{1cm}

%% Inicio do artigo %%
\section{Introdução}
O método de Jacobi-Richardson é um algorítimo iterativo usado para achar soluções aproximadas para sistemas de equações lineares que possam ser representados através de matrizes diagonalmente dominantes.

Esse método é particularmente interessante para ser analisado, pois possui grande potencial de paralelização, uma vez que cada iteração depende apenas dos dados da anterior, e as linhas são independentes.

A proposta é criar um algorítimo para executar o método de Jacobi-Richardson de maneira eficiente, paralelizando os cálculos onde possível.
\pagebreak

\section{Algorítimo}
Dado o seguinte sistema linear: $Ax=B$, o vetor $x$ pode ser aproximado (iterativamente) usando a seguinte recorrência:
\begin{align}
\label{eq:rec}
x^{(k)} = D^{-1}(b - Rx^{(k-1)})
\end{align}

Onde $D$ é uma matriz contendo apenas a diagonal de $A$, e $R = A - D$. Para a primeira iteração podem ser adotados valores arbitrários para $x^{(0)}$.

A partir dessa recorrência podemos determinar a seguinte fórmula para calcular cada elemento de $x^{(k)}$:
\begin{align}
\label{eq:elem}
x^{(k)}_i = \frac{1}{a_{ii}}\bigg[b_i - \sum_{i\neq j}a_{ij}x^{(k-1)}_j\bigg]
\end{align}

\section{Análise do Código (github.com/bardes/T1-CON)}

Como cada linha da matriz pode ser calculada concomitantemente, a função \emph{comput\_line} foi criada para calcular cada elemento de $x^{(k)}$ de maneira independente. Chamar essa função uma vez para cada linha equivale a uma iteração do algorítimo.

\begin{lstlisting}
static double compute_line(size_t line, const jacobi_matrix *m,
                           gsl_vector *work_area)
{
    gsl_vector_const_view l = gsl_matrix_const_row(m->A, line);
    mul_elements(work_area, &l.vector, m->x);
    return (gsl_vector_get(m->B, line) - sum_elements(work_area)) /
            gsl_vector_get(m->diag, line);
}
\end{lstlisting}

Essa implementação essencialmente adapta a Formula \ref{eq:elem} para uma função C. Vale notar que a função recebe como argumento um \emph{buffer} onde são guardados os resultados parciais. Esse buffer é passado como argumento para evitar \emph{overhead} de múltiplas alocações de memória para cada iteração.

A função \emph{compute\_block} é usada para dividir o trabalho entre múltiplas \emph{threads} que computam blocos de linhas em paralelo.

\begin{lstlisting}
static void *compute_block(void *tdata)
{
    thread_data *d = tdata;
    for(size_t line = d->from; line < d->to; ++line)
        d->result->data[line] = compute_line(line, &d->m, d->work_area);
    pthread_exit(NULL);
}
\end{lstlisting}

O loop principal é bastante simples:

\begin{lstlisting}
double err = 0; size_t itrs;
for(itrs = 0; itrs < max_iter; ++itrs) {

    // Dispara cada thread
    for(size_t i = 0; i < n_threads; ++i)
        if(pthread_create(threads + i, NULL, compute_block, td + i))
            FATAL_MSG(0, EXIT_FAILURE, "Failed to create thread!");

    // Espera os resultados
    for(size_t i = 0; i < n_threads; ++i)
        pthread_join(threads[i], NULL);

    // Copia para o vetor de resultados
    gsl_vector_memcpy(m.x, tmp);

    // Verifia se o erro esta dentro do limite
    err = check_error(test_row, &td[0].m, tmp);
    if(fabs(err) < max_err) break;
}
\end{lstlisting}

A cada iteração o número de \emph{threads} especificado pelo usuário é criado e o trabalho dividido entre elas. Após o término de todas as \emph{threads} o resultado parcial ($tmp$) é movido para o vetor $x$ e o erro é verificado.

\section{Resultados}
Os testes foram executados em uma máquina x86\_64 com 4 núcleos de 1GHz. Cada matriz foi testada 10 vezes seguidas, calculando-se o tempo médio e o desvio padrão dos tempos.

\begin{table}[h!]\centering
	\pgfplotstabletypeset[col sep=tab]{r1}
	\caption{Dados coletados com 1 thread.}
\end{table}

Aqui fica claro que a complexidade do algorítimo é $O(n^2)$ em relação a ordem da matriz, o que é esperado, já que esse é de fato o número de elementos de uma matriz quadrada.

Vale a pena notar também que o desvio padrão também cresce conforme a ordem da matriz, o que novamente é esperado, já que quanto mais tempo a matriz leva para ser processada, maior a quantidade de eventos externos que alteram o tempo de execução, como por exemplo outros processos ou mesmo o próprio escalonador do S.O.

\begin{table}[h!]\centering
	\pgfplotstabletypeset[col sep=tab]{r2}
	\caption{Dados coletados com 2 threads.}
\end{table}

Ao adicionar uma segunda thread os ganhos de performance são notáveis. As matrizes de ordem menor mostram um ganho menor, porém ainda significativos. Usando a matriz de ordem 4000 como referência podemos calcular o \emph{speedup}: $1-\frac{1196.04}{733.94} \approx 0.62961 = 62.96\%$. Comparando-se esse valor ao valor ótimo de 100\% nota-se que uma porção do desempenho da versão paralela é perdida devido a \emph{overheads} causados tanto pela aplicação quanto pelo S.O. para lidar com as threads.

\begin{table}[h!]\centering
	\pgfplotstabletypeset[col sep=tab]{r4}
	\caption{Dados coletados com 4 threads.}
\end{table}
Com 4 threads os ganhos já não são mais tão grandes, na verdade para matrizes pequenas a execução fica {\bf mais lenta} do que com 2 threads. Ainda assim para as matrizes maiores os ganhos ainda aparecem.

Nesse caso o \emph{speedup} fica bastante abaixo do valor ótimo de 400\%: $1-\frac{1196.04}{665.18} \approx 0.79806 = 79.81\%$. Essa diferença grande pode ser explicada pelas limitações de hardware, já que com 4 threads a carga de CPU fica superior ao número de núcleos, forçando o escalonamento constante de tarefas, o que diminui bastante a eficiência da aplicação.

\section{Conclusão}
Os dados coletados mostram que de fato o método de Jacobi-Richardson pode ter ganhos de performance utilizando-se de técnicas de programação concorrente, porém é necessário medir e estudar esses ganhos, já que múltiplos fatores como escalonador, carga do sistema, número de núcleos, etc... influenciam a eficácia da implementação paralela do algorítimo.

%% Bibliografia %%
\begin{thebibliography}{5}
\bibitem{ref1}
Ron Larson, David C. Falvo
\\\textit{Elementary Linear Algebra, 6th Edition} ISBN-13: 9780618783762
 
\bibitem{ref2} 
Marina Andretta
\\\texttt{http://www.icmc.usp.br/~andretta/ensino/aulas/sme0301-1-10/
\\sistemaslinearesiterativosjacobirichardson.pdf}
\end{thebibliography}

\end{document}